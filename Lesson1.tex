\documentclass{article}
\usepackage[utf8x]{inputenc}
\usepackage[vietnamese]{babel}


\begin{document}

\section{Trí tuệ là gì?}
Trí tuệ là khả năng học, tiếp thu, nhận thức, sáng tạo và vận dụng vào thực tiễn. \\[0.2cm]
Theo Howard Gardner, có 9 dạng thức thông minh: \\[0.2cm]
- Trí thông minh Tự nhiên \\ [0.1cm]
- Trí thông minh Âm nhạc và Thính giác \\[0.1cm]
- Trí thông minh Toán học và Logic \\[0.1cm]
- Trí thông minh Triết học \\[0.1cm]
- Trí thông minh Tương tác và Giao tiếp \\[0.1cm]
- Trí thông minh Thể chất \\[0.1cm]
- Trí thông minh Ngôn ngữ \\[0.1cm]
- Trí thông minh Nội tâm \\[0.1cm]
- Trí thông minh Không gian và Thị giác

\section{Trí tuệ nhân tạo là gì? Những tiến bộ nào góp phần dẫn đến cuộc cách mạng AI?}
Trí tuệ nhân tạo là trí tuệ của máy tính.\\[0.2cm]
Những tiến bộ góp phần dẫn đến cuộc cách mạng AI: Ứng dụng sâu rộng gần\\[0.1cm] như trong mọi lĩnh vực như Vision, Robotics, Healthcare, Business Intelligence\\[0.1cm] and Analytics, Ad, Sale, CRM, ... và sự phát triển của Deep Learning, Big Data

\section{Học là gì? Học máy là gì? Kiến thức kĩ năng có thể được biểu diễn trong máy tính ra sao? Vị trí của ML trong AI}
Theo như trong bài giảng:\\[0.2cm]
1. Học là thu thập kiến thức, kỹ năng mới thông qua trải nghiệm, giáo dục, nghiên cứu\\[0.1cm]
2. Học máy là máy tính tự động học qua các trải nghiệ\\[0.1cm]
3. Kiến thức kĩ năng có thể được biểu diễn trong máy tính qua hàm ẩn tối ưu tức là mô hình hóa input qua một hàm số từ đó đưa ra output. Quá trình đó dựa trên việc thí nghiệm, phân tích dữ liệu, học và báo cáo kết quả.\\[0.2cm]
 ML là phương tiện để chinh phục mục AI.

\section{TEFPA cần được mô tả và cung cấp để máy tính tự học cách giải quyết một tác vụ là gì?}
TEFPA giúp giải quyết bài toán tối ưu.

\section{Mô tả các ứng dụng}
a, Máy tính chuyển 1 tấm ảnh kém chất lượng lên thành ảnh rõ nét\\[0.2cm]
- Tác vụ đầu vào: Ảnh kém chất lượng.\\[0.1cm]
- Tác vụ đầu ra: Ảnh rõ nét.\\[0.2cm]
- Kinh nghiệm: Xem rất nhiều ảnh kém chất lượng và ảnh rõ nét.\\[0.2cm]
- Cách thức đánh giá: Độ trùng khớp giữa ảnh chuẩn và ảnh phục hồi.
\\[0.3cm]
 b, Máy tính xử lí ảnh chụp X-quang và dự đoán bệnh\\[0.2cm]
 - Tác vụ đầu vào: Ảnh chụp X-quang.\\[0.1cm]
 - Tác vụ đầu ra: Dự đoán bệnh.\\[0.2cm]
 - Kinh nghiệm: Xem rất nhiều ảnh chụp X-quang và bệnh tương ứng.\\[0.2cm]
 - Cách thức đánh giá: Độ trùng khớp giữa bệnh dự đoán và bệnh thực tế
 \\[0.3cm]
 c, Máy tính đọc một email của khách hàng và tự chuyển đến thư mục tương ứng như "cảm ơn", "khiếu nại",...\\[0.2cm]
 - Tác vụ đầu vào: email khách hàng.\\[0.1cm]
 - Tác vụ đầu ra: Chuyển đến thư mục tương ứng.\\[0.2cm]
 - Kinh nghiệm: Xem rất nhiều email và thư mực tương ứng.\\[0.2cm]
 - Cách thức đánh giá: Độ trùng khớp giữa email và thư mục.
\section {Ý nghĩa câu phát biểu: Máy tính "học" bằng cách tìm kiếm trong không gian các hàm số (chương trình máy tính)}
Các hàm số được lập trình trên máy tính nên máy tính sẽ tìm kiếm các hàm số này trong không gian máy tính.

\section{Hai vấn đề chính về không gian hàm mà ta cần đặc biệt chú ý để giúp máy tính tự tìm kiếm hàm có độ khái quát hoá cao}
- Thông qua TEP để tìm kiếm trong không gian hàm.\\[0.2cm]
- Hàm cần tìm phải khá giống hàm ẩn tối ưu thông qua chuẩn đánh giá.

\end{document}
