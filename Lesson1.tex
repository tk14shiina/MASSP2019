\documentclass{article}
\usepackage[utf8x]{inputenc}
\usepackage[vietnamese]{babel}

\begin{document}

\section{Trí tuệ là gì?}
Trí tuệ là khả năng học, tiếp thu, nhận thức, sáng tạo và vận dụng vào thực tiễn. \\[0.2cm]
Theo Howard Gardner, có 9 dạng thức thông minh: \\[0.2cm]
- Trí thông minh Tự nhiên \\ [0.1cm]
- Trí thông minh Âm nhạc và Thính giác \\[0.1cm]
- Trí thông minh Toán học và Logic \\[0.1cm]
- Trí thông minh Triết học \\[0.1cm]
- Trí thông minh Tương tác và Giao tiếp \\[0.1cm]
- Trí thông minh Thể chất \\[0.1cm]
- Trí thông minh Ngôn ngữ \\[0.1cm]
- Trí thông minh Nội tâm \\[0.1cm]
- Trí thông minh Không gian và Thị giác

\section{Trí tuệ nhân tạo là gì?}
Trí tuệ nhân tạo là trí tuệ của máy tính.

\section{Học là gì? Học máy là gì? Kiến thức kĩ năng có thể được biểu diễn trong máy tính ra sao?}
Theo như trong bài giảng:\\[0.2cm]
- Học là thu thập kiến thức, kỹ năng mới thông qua trải nghiệm, giáo dục, nghiên cứu\\[0.2cm]
- Học máy là máy tính tự động học qua các trải nghiệm[0.1cm]
- Kiến thức kĩ năng có thể được biểu diễn trong máy tính qua hàm ẩn tối ưu tức là mô hình hóa input qua một hàm số từ đó đưa ra output. Quá trình đó dựa trên việc thí nghiệm, phân tích dữ liệu, học và báo cáo kết quả.
\end{document}
